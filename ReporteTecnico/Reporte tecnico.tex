\documentclass[12pt]{article}
\usepackage[utf8]{inputenc}
\usepackage{graphicx}
\graphicspath{ {images/} }

%opening
\title {Reporte técnico}
\author{Equipo TSCRUM}
\date{09/05/2018}

\begin{document}

\maketitle

\section{Arquitectura y web services}
\includegraphics[scale=0.4]{Diagrama}



La arquitectura del proyecto se encuentra dividida en varias capas, esto con el fin de aplicar prácticas que todo proyecto de software requiere a nivel de producción. El sistema cuenta con las siguientes capas:
Usuario.
Presentación.
Servicios.
Negocio.
Datos.
Recursos.


\section{Capa de usuario}
Esta capa repesenta la forma en que el usuario accede al sistema, para ello basta ingresar a través de un navegador web al sitio de TSCRUM. \newline\newline

\section{Capa de presentación}
Esta capa se encuentra dentro de un contenedor docker, aquí se emplearon las tecnologías HTML5, CSS3, Bootstrap, Java Script y Angular.\newline\newline

\section{Capa de servicios}
Contenida en el segundo contenedor docker, esta capa está diseñada para poder trabajar mediante protocolo HTTP o HTTPS.\newline\newline

\section{Capa de negocios}
Contenida en el segundo contenedor docker, emplea las tecnologías NodeJS, npm y Sequelize.\newline\newline

\section{Capa de datos}
Contenida en el segundo contenedor docker, aquí se encuentran dos REST APIS, la correspondiente a funciones de miembros y la correspondiente a proyectos.\newline\newline

\section{Capa de recursos}
Esta capa encontramos el sistema gestor de bases de datos PostgreSQL, el cual se encarga de alamacenar toda la información correspondiente a miembros y proyectos del sistema.\newline\newline

\section{Fuentes de datos}
Aquí encontramos las diversas formas que las capas usan para comunicarse entre si, como lo son los Tokens de seguridad, mensajes tipo JSON, peticiones GET/POST y comunicación por protocolo TCP/IP.\newline\newline

\end{document}
